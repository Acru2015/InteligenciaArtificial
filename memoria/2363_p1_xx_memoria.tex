\documentclass{article}
\usepackage{listings}

\title{Memoria P1 Anton Oellerer}          
\date{2017-10-06}         
\author{Anton Oellerer}   

\begin{document}
\maketitle
\section{Simultad coseno recursivo}
\subsection{Pseudoc\'odigo}
\begin{itemize}
	\item \textbf{Entrada:} vectores x, y
	\item \textbf{Salida:} la simultud coseno de de los vectores, calculado recursivo
	\item \textbf{Procesiamento:} Si los vectores estan validos (no nil y no solo 0s) calcula el productode los vectores recursivo. Despues dividie el producto por el producto de los raizes de los cuadrados de los vectores.
\end{itemize}
\subsection{Ejemplos}
\begin{lstlisting}
(sc-rec nil nil) -> nil
(sc-rec (0 0) (1 2)) -> nil
(sc-rec (1 1) (1 1)) -> 1 
(sc-rec (-1 -1) (1 1)) -> -1 
(sc-rec (1 0) (0 1)) -> 0 
(sc-rec (5 7) (4 3)) -> 0.95 
\end{lstlisting}
\subsection{Comentarios}
\begin{itemize}
	\item A causa de errores de redondear, puede ser que los resultos no son exactos (por ejemplo 1.0000001 en lugar de 1)
	\item La recursion multiplica los primeros dos elementes de los vectores y recurso en el resto hasta no hay mas y en el camino de vuelta summa los productos
	\item Cuando un vector est nil o solo de 0s, nil esta retonado
	\item Los cuadrados son calculado por el producto de el vector con su mismo
\end{itemize}
\section{Simultad coseno con mapcar}
\subsection{Pseudoc\'odigo}
\begin{itemize}
	\item \textbf{Entrada:} vectores x, y
	\item \textbf{Salida:} la simultud coseno de de los vectores, calculado con mapcar
	\item \textbf{Procesiamento:} Si los vectores estan validos (no nil y no solo 0s) calcula el productode los vectores con mapcar. Despues dividie el producto por el producto de los raizes de los cuadrados de los vectores.
\end{itemize}
\subsection{Ejemplos}
Como recursivo
\subsection{Comentarios}
\begin{itemize}
	\item A causa de errores de redondear, puede ser que los resultos no son exactos (por ejemplo 1.0000001 en lugar de 1)
	\item El producto de los vectores es calculado por una reduccion con '+' sobre un mapcar que multiplica todas las parejas de los vectores
	\item Cuando un vector est nil o solo de 0s, nil esta retonado
	\item Los cuadrados son calculado por el producto de el vector con su mismo
\end{itemize}
\section{sc-conf}
\subsection{Pseudoc\'odigo}
\begin{itemize}
	\item \textbf{Entrada:} vector cat, vector de vectores vs, float conf
	\item \textbf{Salida:} Vectores cuya simultad a cat es superior a conf, ordenados
	\item \textbf{Procesiamento:} Por cada vector de vs, calcula la simultad coseno con cat, despues excluye los resultos que son mas peque\~nos que la confianza y ordena la lista
\end{itemize}
\subsection{Ejemplos}
\begin{lstlisting}
(sc-conf '(1 2 3) '((1 2 3) (0 1 0) (4 5 6)) 0.7) -> (0.97463 0.99999)
\end{lstlisting}
\subsection{Comentarios}
\begin{itemize}
	\item Como con los otros, errores de redondear eran posible
\end{itemize}
\section{sc classifier}
\subsection{Pseudoc\'odigo}
\begin{itemize}
	\item \textbf{Entrada:} vectores de vectores cats, texts; funci\'on func
	\item \textbf{Salida:} Por cado texto, la categoria que tiene el simultad coseno maximo y ese resultado)
	\item \textbf{Procesiamento:} Por cada vector de texto, calcula todas las scs con la funcion indicada por las categorias y retorna el identificador y el resulto de la categoria con sc maxima.
\end{itemize}
\subsection{Ejemplos}
\begin{lstlisting}
(sc-classifier '((0 1 2 3 4) (1 3 2 2 0) (2 3 5 8 0)) '((0 1 2 3 4) (1 3 2 1 0) (2 1 1 1 1)) 'sc-mapcar) -> ((0 . 0.99999994) (1 . 0.97230554) (0 . 0.9128709))
\end{lstlisting}
\subsection{Comentarios}
Como puede observar, calcular las scs es mas rapido con mapcar, pero necesita mas espacio.
\end{document}
